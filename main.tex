\documentclass[a4paper,12pt]{article}

\usepackage{hyperref}

\hypersetup{
    hidelinks,
    colorlinks=false,
}

\usepackage[
    type={CC},
    modifier={by-sa},
    version={4.0},
]{doclicense}
\usepackage{authblk}


\title{Why do civilizations collapse? Is our civilization in danger?}

\author[1]{Jiankun Su}
\affil[1]{No.2 High School of East China Normal University}

\date{\today}

\bibliographystyle{unsrt}

\begin{document}
    \maketitle
    \begin{center}
        \doclicenseText
    \end{center}
    \tableofcontents
    \newpage
    
    \section{Introduction}
    It is commonly accepted that a civilization collapses for multivariate reasons, but such reasons can be unified under the perspective of the theory of meme, which makes it possible to involve biological methodologies in historical study.
    
    \section{Meme}
    Every civilization can be regarded as a meme (an idea, behavior, or style that spreads by means of imitation from person to person and often carries symbolic meaning representing a particular phenomenon or theme) \cite{2016_TheSelfishGene}.
    Analogous to a living species, a civilization attains energy with people's contribution to it, both cultural and technological, strengthening its power against harsh environment and discord respectively, like species consuming energy at trophic levels.
    It reproduces by increasing its recognition among a wider range of crowd, known as the spread of a civilization.
    At the same time, all the external factors --- natural or anthropogenic --- sum up the biophysical environment where the civilization lives in.
    In such meaning, a civilization acts like a living creature in various aspects, which makes many biological theories applicable.
    Similar to the fact that a creature becomes extinct due to loss of ecological niche, it can be inducted from history that most civilization collapse because of loss of its ``ecological niche'' --- how it adapt to a specific era.
    
    \section{Ecological niche}
    In ecology, a niche is the match of a species to a specific environmental condition \cite{2015_TheEcoNiche}.
    A species occupies and thrives the niche often because it reaches a local optimum in the efficiency of the energy flowing process, which blocks its competitors out.
    Generally, the loss of ecological niche may be because:
    \begin{itemize}
        \item The niche has vanished as environment changes; or
        \item A more energy-efficient competitor has replaced the original species.
    \end{itemize}
    Such reasons are sound applied to the collapse of a civilization and explain the natural experiments observed well.
    
    The civilization based on Confucianism ideology had ruled China and surrounding regions for millenniums but rapidly collapsed in the 20th century, when the long-lasted family-based agricultural mode of production was ruined by industrialization and urbanization.
    Those environmental changes dissolved its ecological niche --- nobody would need its social functions any more, therefore it is not surprising that Confucianism lost all its believers and faded in the new era.
    This phenomenon reaches an extremity when it comes to the Easter Island.
    The island used to be formerly forested, with more habitable climate conditions than today, and the residents were considered to be the only Polynesians who had invented a writing system and built hundreds of giant statues.
    However, due to environmental degradation led by human activities \cite{1991_ClimaticEasterIsland}, wood, which was a critical means of production for the residents, had once become completely unavailable, causing its population to have dropped to less than $\frac 1 4$ of the peak by the time Europeans arrived \cite{2005_TheRapeOfRapaNui}, with the collapse of a complex social structure and retrogression of technology.
    
    Also, a civilization's ecological niche could be taken by another one, sometimes along with conflicts or revolutions.
    Take the Christian religion and the hellaenic polytheism religions before as an example.
    Instead of being generally limited in the aristocrat class's affairs, the newly-born Christianity provided a vision of salvation for everyone.
    In this way, Christian religion reached a higher ``energy efficiency'' by getting support from a larger number of believers \cite{2018_ChristianCareForPoor}.
    As a result, Christianity received much more resource on the ecological niche of religion than ancient polytheist ones and continued to thrive, even under the persecution of the Roman authority, while disrupting its competitors' niche and eventually made them on the wane.
    Other examples include the Aztec when colonists superseded the maladaptive primitive cultivation technologies and backward social patterns \cite{2016_WhyCivFail}.
    
    It may be doubted that a civilization could also be destroyed by an external might, but observation tells that military conquest alone rarely collapses a civilization.
    A well-known counter-example is the Mongolian Empire: despite epic victories, none of the vanquished has disappeared in history, since the Mongolians could neither make any changes to the geographical environment in which those civilizations lived, nor outweigh and replace them with their nomadic-styled culture which is far less energy-efficient in places other than the grasslands of inner Asia.
    In the east, the Mongolians who had to adapt to the local Han-Chinese culture inevitably spitted up with other relatives \cite{2002_MongolCulture}. 
    The same thing happened to Mongolians across Eurasia and eventually led to the collapse of the empire.
    As long as the ecological niche remains, a civilization will not collapse and eventually expel the invaders. 
    
    In conclusion, the major reason why a civilization collapses is the loss of its ecological niche, resulted by either the disappearance of the original niche or a stronger candidate's replacement.
    
    \section{In danger}
    Based on the view above, our civilization --- the industrialized and globalized modern civilization --- is in a sense in danger.
    This is because the ecological niche our civilization is on could not eternally exist, and, considering the recent blooming technologies, may shortly be eradicated by the coming technological revolutions considered to be on the move.
    For example, the rapid development of AI in exactly recent years has already reshaped standards of industries such as painting, programming, and even academic researches.
    The current social formation is forged by the industrial revolution (specifically, the first industrial revolution), which led to current social division of labor that has been proved efficient by modern history, and such formation is evidently fragile facing impact brought by new technologies since it would be likely that those new-era technologies require starkly different organizations of society to maximize the profit \cite{2019_ImpactAIOnLabor}, not to mention other external threats like environment issues or extremist ideology here.
    Given the history, it could not be ignorantly believed that our civilization will not encounter danger.
    
    \section{Evolution}
    However, like living species, a civilization not only dies out, but also evolves.
    Evolution of a civilization happens when a civilization reforms under an external selective pressure, such as environmental change or social development, and is usually accompanied by the collapse of the old-day civilization and the emergence of the new.
    Science tells that the descendant grouping necessarily contain some of the genetic segments of their ancestors.
    And the same in biological evolution happens for a civilization.
    It is same in history that collapsed civilizations usually will not disappear into complete void, but leave some of its adaptable characters instead, which are usually assimilated by followers.
    For example, the achievements of Roman civilization were not ruined with the collapse of Rome.
    On the contrary, such achievements persisted and spread across regions which had previously been viewed barbaric.
    Modern Europeans can not deny the fact that they are successors of the Roman civilization, which is still alive and influencing the modern society.
    This provides a more optimistic perspective on the collapse of civilizations, since most of them do not actually become extinct, but somehow continue to exist through a fairly long time period.
    It is the same back to our civilization.
    Even though we have to admit that social development may cause its decline, we can still be proud of the fact that some of our civilization's achievements will be inherited by future generations and contribute to humanity.
    
    \bibliography{ref.bib}
\end{document}